% chktex-file 44
\renewcommand{\thesection}{\Alph{section}}
\section{Purposes and Goals}\label{sec:purposeandgoals}
\subsection{Purpose}\label{subsec:purpose}
Internships provide students with a valuable opportunity to apply their skills in real job environments while enabling companies to connect with 
fresh talent. However, the process of finding and securing internships can be challenging for both parties.

Students\&Companies (S\&C) is a platform designed to facilitate this connection throughout the internship process. It allows 
students to match their preferences with available opportunities, ensuring internships align with their experiences and skills. 
Companies can specify project requirements to attract suitable candidates.

The platform supports both students and companies in two phases: recommendation and selection. It utilizes keyword searches 
and statistical analyses to evaluate internship and student data, ensuring a match that meets the needs of both parties. 
Students can search for internships and receive notifications about appealing opportunities, while companies can publish offers and get 
alerts when student CVs match their criteria. Once mutual interest is established, the selection phase begins, where the platform assists 
with interviews and finalizes the process. It also monitors the internship journey, providing the possibility to exchange feedback and 
enabling direct communication to address any questions. Additionally, the universities of the students can oversee the process to ensure 
that everything is going smoothly.

\subsection{Goals}\label{subsec:goals}
[G1] All unregistered Users must be able to create an account on the platform using their specific email address and log in to the platform.

[G2] Students must be able to upload their CVs on the platform.

[G3] Students must be able to search for available internship offers on the platform.

[G4] Companies must be able to publish internship offers on the platform.

[G5] Users must receive notifications about relevant events.

[G6] Students and Companies must receive recommendations based on statistical analyses.

[G7] Students must be able to proactively apply for internships, before the submission deadline.

[G8] Companies must be able to set up the interview process.

[G9] During the interview process, students must be able to respond to the company's questions through questionnaires.

[G10] At the end of the interview process, companies must be able to collect the students' responses.

[G11] Companies must be able to send the results of the interviews to the students.

[G12] Students must be able to accept or reject the internship offer after receiving the interview results.

[G13] Students and companies must be able to write feedback regarding their internship experiences.

[G14] Students and companies must be able to communicate with each other during internship process.

[G15] Universities must be able to monitor the internship process of their students.


\newpage
\section{Scope}\label{sec:scope}
\subsection{Scope}\label{subsec:scope}
Students\&Companies (S\&C) aims to provide the best matching service between students and companies for internships. The platform will be available
to students, companies, and universities.

If it is their first time using the platform, Students looking for internships can create an account using their educational institution's email and 
select their university from a list of the ones that collaborate with the platform. 
After creating an account, they need to fill out their profiles to describe their skills, experiences, and preferences. To complete
their profiles, they must also upload their CVs.
Otherwise they can log in to the platform using their credentials.

The universities associated with the students through their educational emails will be notified about the students' registration on the platform. 
The system will then add the registered students to the university’s list, allowing the universities to monitor their internship activities.

Companies wishing to announce internship opportunities can create an account on the platform, if they do not already have one. They need to
provide the necessary information for the internship announcement such as the job description, requirements, and the number of interns needed etc. 
Companies can also specify the skills they are looking for in students.

The platform will extract key information from the students' profiles and the companies' internship announcements, along with historical feedback and 
information collected from previous internships, to recommend the best matches for both parties. Students will receive notifications about 
recommended internships, and companies will be alerted when existing students with matching CVs meet their needs.

Once students apply for the offers they are interested in, the companies will receive the applications and can select the students they wish 
to interview. The platform will assist companies in setting up the interview process and will allow them to record and store students' responses 
using the platform. Students will be able to respond to companies' questions through the tools or channels provided.

At the end of the interviews, companies can send the results to the students. Upon receiving the results, students can decide whether to accept 
or reject the offer, and the companies will be notified of their decisions. If a student accepts the offer and starts the internship, the platform 
will update the internship activities about that student at their universities. Students and companies can use the channels provided by the platform to
communicate with each other during the internship, and also they can provide feedback and complains regarding their experiences in dedicated section
which can be monitored by the students' universities. 

\subsection{Phenomena}
Referring to the Jackson-Zave distinction between the world and the machine in the context of the S\&C platform, the following phenomena are 
identified, specifying which parts are controlled by the machine and which parts are controlled by the world, shown in table~\ref{table:phenomena}.

\begin{table}[H]
    \caption*{\textbf{Phenomena based on the Jackson-Zave model}}
    \centering 
    \begin{tabular}{|c|p{24em}|c|c|}
    \hline
    \rowcolor{bluepoli!40} % comment this line to remove the color
    \small\textbf{Code} & \small\textbf{Phenomenon} & \small\textbf{Shared} & \small\textbf{Control} \T\B \\
    \hline
    \small\textbf{P1} &\small \textbf{User registration} & Yes & World \T\B \\
    \small\textbf{P2} &\small \textbf{User login} & Yes & World \T\B\\
    \small\textbf{P3} &\small \textbf{Check username and password} & No & Machine \T\B\\
    \small\textbf{P4} &\small \textbf{Student creates CV using text editor} & No & World  \T\B \\
    \small\textbf{P5} &\small \textbf{Student uploads CV in profile} & Yes & World  \T\B \\
    \small\textbf{P6} &\small \textbf{Student updates profile information} & Yes & World  \T\B \\
    \small\textbf{P7} &\small \textbf{Student searches available internships} & Yes & World \T\B\\
    \small\textbf{P8} &\small \textbf{Company publishes internship offers} & Yes & World \T\B \\
    \small\textbf{P9} &\small \textbf{Platform notifies users that a deadline has expired} & Yes & Machine \B\\
    \small\textbf{P10} &\small \textbf{Platform suggests recommendations} & Yes & Machine \T\B \\
    \small\textbf{P11} &\small \textbf{Platform adds student to university's list} & No & Machine \T\B \\
    \small\textbf{P12} &\small \textbf{A student submits an application for an internship} & Yes & World \B\\
    \small\textbf{P13} &\small \textbf{Company selects candidates to interview} & Yes & World \T\B\\
    \small\textbf{P14} &\small \textbf{Student participates to the interview} & Yes & World \T\B \\
    \small\textbf{P15} &\small \textbf{Company sends interview results} & Yes & World \B\\
    \small\textbf{P16} &\small \textbf{Student accepts or rejects the internship} & Yes & World \T\B \\
    \small\textbf{P17} &\small \textbf{ST and CO write feedback on the internship} & Yes & World \T\B\\
    \small\textbf{P18} &\small \textbf{ST and CO view feedback on the internship} & Yes & World \B\\
    \small\textbf{P19} &\small \textbf{Student views the offers' description} & Yes & World \T\B \\
    \small\textbf{P20} &\small \textbf{Company views the students' profile} & Yes & World \T\B\\
    \small\textbf{P21} &\small \textbf{University views the list of its students} & Yes & World \B\\
    \small\textbf{P22} &\small \textbf{University tracks the internship process} & Yes & World \T\B\\
    \hline
    \end{tabular}
    \\[10pt]
    \caption{Phenomena in the S\&C context}\label{table:phenomena}
\end{table}

\section{Definitions, Acronyms, Abbreviations}\label{sec:definitions}
\subsection{Definitions}\label{subsec:definitions}
\begin{itemize}
    \item \textbf{Student:} A person who is looking for internships.
    \item \textbf{Company:} An organization which wants to announce internship opportunities to students.
    \item \textbf{University:} An educational institution that is related to students and their internships.
    \item \textbf{User:} A generic term for students, companies, and universities who use the platform.
    \item \textbf{Candidate:} A term for students whose applications are selected and that will take part in the interview process.
    \item \textbf{Internship:} A opportunity offered by companies to students to gain practical experience in a real job environment.
    \item \textbf{CV:} Curriculum Vitae, a document that contains all necessary information about students to be able to apply for internships.
    \item \textbf{Recommendation:} A suggestion made by the platform to students and companies based on statistical analyses and keyword searches.
    \item \textbf{Selection:} The process of choosing students and companies to process the interview.
    \item \textbf{Feedback:} Helpful information written by students and companies about their internship experiences to improve a performance 
    of the two parties.
    \item \textbf{Notification:} A message sent by the platform to inform students and companies about important events, such as new internship 
    offers, matching CVs, interview results etc.
    \item \textbf{Interview:} A meeting between students and companies to decide an assignment of the internship offer.
    \item \textbf{Platform:} The Students\&Companies (S\&C) system that provides the services to students, companies, and universities about
    internships.
    \item \textbf{Keyword:} A significant word or tag used to describe content, such as the skills, experiences, and preferences of students and 
    companies.
    \item \textbf{Comment:} The text that is written by students and companies to provide feedback or complaints about their internship experiences.
    \item \textbf{Complain:} A text that expresses dissatisfaction, issues, or annoyance about the internship experiences.
\end{itemize}

\subsection{Acronyms}\label{subsec:acronyms}
\begin{itemize}
    \item \textbf{S\&C:} Students\&Companies
    \item \textbf{CV:} Curriculum Vitae
    \item \textbf{UI:} User Interface
    \item \textbf{UX:} User Experience
    \item \textbf{API:} Application Programming Interface
    \item \textbf{HTTP:} Hypertext Transfer Protocol
    \item \textbf{HTTPS:} Hypertext Transfer Protocol Secure
    \item \textbf{TLS:} Transport Layer Security
    \item \textbf{REST:} Representational State Transfer

\end{itemize}

\subsection{Abbreviations}\label{subsec:abbreviations}
\begin{itemize}
    \item \textbf{[Gn]:} Used to number the goals, where Gn is the n-th goal.
    \item \textbf{[Pn]:} Used to number the phenomena, where Pn is the n-th phenomenon.
    \item \textbf{[Rn]:} Used to number the requirements, where Rn is the n-th requirement.
    \item \textbf{[An]:} Used to number the domain assumptions, where An is the n-th assumption.
    \item \textbf{CO:} Company
    \item \textbf{ST:} Student
    \item \textbf{UNI:} University
\end{itemize}

\section{Revision History}\label{sec:revisionhistory}
%%TODO

\section{Reference Documents}\label{sec:reference}
\begin{itemize}
    \item `` The World and the Machine: A model for the functional architecture '' by Michael Jackson and Pamela Zave.
    \item Assignment RDD AY 2024–2025.
\end{itemize}

\section{Document Structure}\label{sec:structure}
This document is structured as follows:
\begin{itemize}
    \item \textbf{Section 1: Introduction} 
    \\It contains the purpose, goals, scope, and phenomena identified in the context of the S\&C project, 
    including the specification of definitions, acronyms, and abbreviations of the terms used in this document. In addition, it notes the revision 
    history for updates to the document and the reference documents that were used during the development of this document.
    \item \textbf{Section 2: Overall Description}
    \\ In this section, the general view of the S\&C project is presented, including the product perspective, functions, characteristics, 
    constraints, assumptions, and dependencies. Through the use of UML diagrams, the statecharts, and the scenarios to explain the project's 
    functionalities.
    \item \textbf{Section 3: Specific Requirements}
    \\ A detailed description of the requirements of the project will be provided in this section. Using the various point of views, the external   
    interface requirements, functional requirements, performance requirements, design constraints, software system attributes, and other 
    requirements. In particular to easier understanding, the use cases, sequence diagrams, mapping tables and simple drafts of the user interface 
    will be presented.
    \item \textbf{Section 4: Formal Analysis}
    \\ Using the Alloy language to model the project and to verify the logical consistency of the model. It will also be used to check the
    correctness of the requirements and to present the main features guaranteed by the platform. Also, a part of the scenarios will be presented 
    by predicates using show and run commands.
    \item \textbf{Section 5: Effort Spent}
    The time spent by each group member on each task will be registered in this section. It will be used to present the effort dedicated 
    by each member and to present the progress of the development of the project.
    \item \textbf{Section 6: References}
    \\ The other references that not include in the reference documents will be added in this section.
\end{itemize}