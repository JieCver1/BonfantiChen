% chktex-file 44
\renewcommand{\thesection}{\Alph{section}}
\section{Purposes and Goals}\label{sec:purposeandgoals}
\subsection{Purpose}\label{subsec:purpose}
Internships provide students with a valuable opportunity to apply their skills in real job environments while enabling companies to connect with 
fresh talent. However, the process of finding and securing internships can be challenging for both parties.

Students\&Companies (S\&C) is a platform designed to facilitate this connection throughout the internship process. It allows 
students to match their preferences with available opportunities, ensuring internships align with their experiences and skills. 
Companies can specify project requirements to attract suitable candidates.

The platform supports both students and companies in two phases: recommendation and selection. During the recommendation phase, 
it utilizes keyword searches and statistical analysis to assess internship information. Students can search for internships and receive 
notifications about appealing opportunities, while companies can publish offers and get alerts when student CVs match their criteria.
Once mutual interest is established, the selection phase begins, where the platform assists with interviews and finalizes the process. 
It also monitors the internship journey, providing feedback and enabling direct communication to address any questions. Additionally, 
the universities of the students can oversee the process to unsure the process is going smoothly.

\subsection{Goals}\label{subsec:goals}
[G1] All unregistered users must be able to create an account on the platform using their specific email address.

[G2] Students must be able to upload their CVs on the platform, containing their skills, experiences, preferences, and other relevant information.

[G3] Students must be able to search for available internship offers on the platform based on their preferences, using a simple keyword search.

[G4] Companies must be able to publish internship offers on the platform, specifying the requirements and job descriptions.

[G5] Students must be able to receive notifications when internships that they may be interested in become available.

[G6] Companies must be able to receive notifications when a student's CV matches their internship offer.

[G7] Students must be able to receive recommendations of companies based on statistical analyses of their profiles and the needs of companies.

[G8] Companies must be able to receive recommendations of students based on statistical analyses of students's profiles and their needs.

[G9] Students must be able to proactively apply for internships they are interested in.

[G10] Companies can select students based on their CVs and other profile information.

[G11] Companies must be able to set up the interview process with the students who have been selected.

[G12] During the interview process, students must be able to respond to the company's questions through questionnaires or other communication tools.

[G13] During the interview process, companies must be able to collect the students' responses.

[G14] Companies must be able to send the results of the interviews to the students.

[G15] Students can accept or reject the offer after receiving the interview results.

[G16] Students and companies must be able to view all feedback and suggestions related to internship experiences.

[G17] Students and companies must be able to write feedback and suggestions regarding their internship experiences.

[G18] Students and companies must be able to communicate with each other through the platform regarding all aspects of the internship process.

[G19] Universities must be able to monitor the internship process to track any events. 


\newpage
\section{Scope}\label{sec:scope}
\subsection{Scope}\label{subsec:scope}
Students\&Companies (S\&C) aims to provide the best matching service between students and companies for internships. The platform will be available
to students, companies, and universities.

If it is their first time using the platform, Students looking for internships can create an account using their educational institution's email and select their university from a list of the ones that collaborate with the platform. 
After creating an account, they need to fill out their profiles with keywords that describe their skills, experiences, and preferences. To complete
their profiles, they must also upload their CVs.
Otherwise they can log in to the platform using their credentials.

The universities associated with the students through their educational emails will be notified about the students' registration on the platform. 
The system will then add the registered students to the university’s list, allowing the universities to monitor their internship activities.

Companies wishing to announce internship opportunities can create an account on the platform, if they do not already have an account. They need to
provide the necessary information for the internship announcement such as the job description, requirements, and the number of interns needed etc. 
Companies can also specify the keywords that describe the skills they are looking for in students.

The platform will use the keywords in the students' profiles and the companies' internship announcements, along with historical feedback and 
information collected from previous internships, to recommend the best matches for both parties. Students will receive notifications about 
recommended internships, and companies will be alerted when existing students with matching CVs meet their needs.

Once students apply for the offers they are interested in, the companies will receive the applications and can select the students they wish 
to interview. The platform will assist companies in setting up the interview process and will allow them to record and store students' responses 
using the platform. Students will be able to respond to companies' questions through the tools or channels provided.

At the end of the interviews, companies can send the results to the students. Upon receiving the results, students can decide whether to accept 
or reject the offer, and the companies will be notified of their decisions. If a student accepts the offer and starts the internship, the platform 
will update the internship activities about students at their universities. Students and companies can use the channels provided by the platform to
communicate with each other during the internship, and at the end, they can provide feedback and suggestions regarding their experiences. Meanwhile,
universities can track the internship process and view the messages and information exchanged between students and companies.

\subsection{Phenomena}
Referring to the Jackson-Zave distinction between the world and the machine in the context of the S\&C platform, the following phenomena are 
identified, specifying which parts are controlled by the machine and which parts are controlled by the world, shown in table~\ref{table:phenomena}.

\begin{table}[H]
    \caption*{\textbf{Phenomena based on the Jackson-Zave model}}
    \centering 
    \begin{tabular}{|c|p{20em}|c|c|}
    \hline
    \rowcolor{bluepoli!40} % comment this line to remove the color
    \textbf{Code} & \textbf{Phenomenon} & \textbf{Shared} & \textbf{Who controls it} \T\B \\
    \hline
    \textbf{P1} & \textbf{User registration} & Yes & World \T\B \\
    \textbf{P2} & \textbf{User login} & Yes & World \T\B\\
    \textbf{P3} & \textbf{Check username and password} & No & Machine \T\B\\
    \textbf{P4} & \textbf{Student creates CV using text editor} & No & World  \T\B \\
    \textbf{P5} & \textbf{Student uploads CV in profile} & Yes & World  \T\B \\
    \textbf{P6} & \textbf{Student update profile information} & Yes & World  \T\B \\
    \textbf{P7} & \textbf{Student searches internship available} & Yes & World \T\B\\
    \textbf{P8} & \textbf{Company publishes internship offers} & Yes & World \T\B \\
    \textbf{P9} & \textbf{Platform notifies users that a deadline has expired} & Yes & Machine \B\\
    \textbf{P10} & \textbf{Platform suggests recommendations} & Yes & Machine \T\B \\
    \textbf{P11} & \textbf{Platform add student in university's list} & No & Machine \T\B \\
    \textbf{P12} & \textbf{Student applies offer} & Yes & World \B\\
    \textbf{P13} & \textbf{Student rejects or accepts internship} & Yes & World \T\B \\
    \textbf{P14} & \textbf{Company selects candidates to interview} & Yes & World \T\B\\
    \textbf{P15} & \textbf{Student participates interview} & Yes & World \T\B \\
    \textbf{P16} & \textbf{Company send interview results} & Yes & World \B\\
    \textbf{P17} & \textbf{Stu\&Comp write feedback of internship} & Yes & World \T\B\\
    \textbf{P18} & \textbf{Stu\&Comp view feedback of internship} & Yes & World \B\\
    \textbf{P19} & \textbf{Student view the offers' description} & Yes & World \T\B \\
    \textbf{P20} & \textbf{Company view the students' profile} & Yes & World \T\B\\
    \textbf{P21} & \textbf{University view the list of its students} & Yes & World \B\\
    \textbf{P22} & \textbf{University track the internship process} & Yes & World \T\B\\
    \hline
    \end{tabular}
    \\[10pt]
    \caption{Phenomena in the S\&C context}\label{table:phenomena}
\end{table}

\section{Definitions, Acronyms, Abbreviations}\label{sec:definitions}
\subsection{Definitions}\label{subsec:definitions}
\begin{itemize}
    \item \textbf{Student:} A person who is looking for internships.
    \item \textbf{Company:} An organization which wants to announce internship opportunities to students.
    \item \textbf{University:} An educational institution that related to students and their internships.
    \item \textbf{User:} A generic term for students, companies, and universities who use the platform.
    \item \textbf{Internship:} A opportunity offered by companies to students to gain practical experience in a real job environment.
    \item \textbf{CV:} Curriculum Vitae, a document that contains all necessary information about students to able to apply for internships.
    \item \textbf{Recommendation:} A suggestion made by the platform to students and companies based on their statistical analysis and simple 
    keyword searches.
    \item \textbf{Selection:} The process of choosing students and companies to process the interview.
    \item \textbf{Feedback:} Comments and suggestions written by students and companies about their internship experiences.
    \item \textbf{Notification:} A message sent by the platform to inform students and companies about important events, such as new internship 
    offers, matching CVs or interview results etc.
    \item \textbf{Interview:} A meeting between students and companies to decide an assignment of the internship offer.
    \item \textbf{Platform:} The Students\&Companies (S\&C) system that provides the services to students, companies, and universities about
    internships.
    \item \textbf{Keyword:} A label or tag that describes the skills, experiences, and preferences of students and companies.

\end{itemize}

\subsection{Acronyms}\label{subsec:acronyms}
\begin{itemize}
    \item \textbf{S\&C:} Students\&Companies
    \item \textbf{CV:} Curriculum Vitae
    \item \textbf{UI:} User Interface
    \item \textbf{UX:} User Experience
\end{itemize}

\subsection{Abbreviations}\label{subsec:abbreviations}
\begin{itemize}
    \item \textbf{[Gn]:} Used to number the goals, where Gn is the n-th goal.
    \item \textbf{Stu\&Comp:} students and companies
\end{itemize}

\section{Revision History}\label{sec:revisionhistory}
%%TODO

\section{Reference Documents}\label{sec:reference}
\begin{itemize}
    \item `` The World and the Machine: A model for the functional architecture '' by Michael Jackson and Pamela Zave.
    \item Assignment RDD AY 2024–2025.
\end{itemize}

\section{Document Structure}\label{sec:structure}
This document is structured as follows:
\begin{itemize}
    \item \textbf{Section 1: Introduction} 
    \\It contains the purpose, goals, scope, and phenomena identified in the context of the S\&C project, 
    including the specification of definitions, acronyms, and abbreviations of the terms used in this document. In addition, it notes the revision 
    history for updates to the document and the reference documents that were used during the development of this document.
    \item \textbf{Section 2: Overall Description}
    \item \textbf{Section 3: Specific Requirements}
    \item \textbf{Section 4: Formal Analysis}
    \item \textbf{Section 5: Effort Spent}
    \item \textbf{Section 6: References}
\end{itemize}