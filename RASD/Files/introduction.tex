
\renewcommand{\thesection}{\Alph{section}}
\section{Purposes and Goals}
\label{sec:purposeandgoals}
\subsection{Purpose}
\label{sec:purpose}
Internships provide students with a valuable opportunity to apply their skills in real job environments while enabling companies to connect with 
fresh talent. However, the process of finding and securing internships can be challenging for both parties.

Students\&Companies (S\&C) is a platform designed to facilitate this connection throughout the internship process. It allows 
students to match their preferences with available opportunities, ensuring internships align with their experiences and skills. 
Companies can specify project requirements to attract suitable candidates.

The platform supports both students and companies in two phases: recommendation and selection. During the recommendation phase, 
it utilizes keyword searches and statistical analysis to assess internship information. Students can search for internships and receive 
notifications about appealing opportunities, while companies can publish offers and get alerts when student CVs match their criteria.
Once mutual interest is established, the selection phase begins, where the platform assists with interviews and finalizes the process. 
It also monitors the internship journey, providing feedback and enabling direct communication to address any questions. Additionally, 
the universities of the students can oversee the process to unsure the process is going smoothly.

\subsection{Goals}
\label{sec:goals}
[G1] Students must be able to search for available internship offers on the platform based on their preferences, using a simple keyword search.

[G2] Companies must be able to publish internship offers on the platform, specifying the requirements and job descriptions.

[G3] Students must be able to upload their CVs on the platform, containing their skills, experiences, preferences, and other relevant information.

[G4] Students must be able to receive notifications when internships that they may be interested in become available.

[G5] Companies must be able to receive notifications when a student's CV matches their internship offer.

[G6] Students must be able to receive recommendations from companies based on statistical analyses of their profiles of students and the needs of companies.

[G7] Companies must be able to receive recommendations from students based on statistical analyses of students's profiles and their needs.

[G8] Students must be able to proactively apply for internships they are interested in.

[G9] After the selection process, companies must be able to set up the interview process with the students who applied.

[G10] After the selection process, students must be able to participate in the interview process with the companies they applied to.

[G11] During the interview process, students must be able to respond to the company's questions through questionnaires or other communication tools.

[G12] During the interview process, companies must be able to collect the students' responses.

[G13] Students and companies must be able to view all feedback and suggestions related to internship experiences.

[G14] Students and companies must be able to write feedback and suggestions regarding their internship experiences.

[G15] Students and companies must be able to communicate with each other through the platform regarding all aspects of the internship process.

[G16] Universities must be able to monitor the internship process to track any events.

[G17] All unregistered users must be able to create an account on the platform using their specific email address.

\newpage
\section{Scope}
\label{sec:Scope}
\subsection{Scope}
Students\&Companies (S\&C) aims to provide the best matching service between students and companies for internships. The platform will be available
to students, companies, and universities.

Students looking for internships can create an account using their educational institution's email if it is their first time using the platform. 
After creating an account, they need to fill out their profiles with keywords that describe their skills, experiences, and preferences. To complete
their profiles, they must also upload their CVs and indicate the official email address of the university overseeing the internship.

The universities associated with the students through their educational emails will be notified about the students' registration on the platform. 
If a university does not have an official account, one will be created automatically, and access credentials will be sent to the appropriate office
email address. The system will then add the registered students to the university’s list, allowing the universities to monitor their internship 
activities.

Companies wishing to announce internship opportunities can create an account on the platform, if they do not already have an account. They need to
provide the necessary information for the internship announcement such as the job description, requirements, and the number of interns needed etc. 
Companies can also specify the keywords that describe the skills they are looking for in students.

The platform will use the keywords in the students' profiles and the companies' internship announcements, along with historical feedback and 
information collected from previous internships, to recommend the best matches for both parties. Students will receive notifications about 
recommended internships, and companies will be alerted when existing students with matching CVs meet their needs.

Once students apply for the offers they are interested in, the companies will receive the applications and can select the students they wish 
to interview. The platform will assist companies in setting up the interview process and will allow them to record and store students' responses 
using the platform. Students will be able to respond to companies' questions through the tools or channels provided.

At the end of the interviews, companies can send the results to the students. Upon receiving the results, students can decide whether to accept 
or reject the offer, and the companies will be notified of their decisions. If a student accepts the offer and starts the internship, the platform 
will update the internship activities about students at their universities. Students and companies can use the channels provided by the platform to
communicate with each other during the internship, and at the end, they can provide feedback and suggestions regarding their experiences. Meanwhile,
universities can track the internship process and view the messages and information exchanged between students and companies.

\section{Definitions, Acronyms, Abbreviations}
\label{sec:definitions}

\section{Revision History}
\label{sec:revision}

\section{Reference Documents}
\label{sec:reference}

\section{Document Structure}
\label{sec:structure}
