\renewcommand{\thesection}{\Alph{section}}
\section{External Interface Requirements}
\subsection{User Interfaces}
As the users of the platform include students, companies, and universities, the user interface is designed to facilitate interaction with 
the platform and meet their needs. For instance, many students may be unfamiliar with the internship search process and may lack experience
in finding job opportunities, so the user interface should be simple and intuitive to use.

In the following section, the principal Graphical User Interfaces (GUIs) are described within drafts, and this part will be further detailed
using refined mockups in the Design Document
\begin{itemize}
    \item \textbf{Welcome page:} The welcome page is the first page users see when they open the platform. It contains the platform's logo 
    and a brief description of its purpose. On this page, users can log in using their credentials or create an account if they are new
    to Students \& Companies.
    
    \item \textbf{Register Page:} During the registration phase, users will first be asked to select their user type (student, company, or
     university). Based on the selected user type, the corresponding registration form will be displayed, and users will be prompted to 
     provide the necessary information to create an account. In addition to common data such as name, email, and password, users will
     need to provide additional information based on their user type. For example, students and companies will be asked to personalize 
     their profiles by adding details such as their field of interest, skills, a short description of themselves, and more to make their
      profiles more appealing to other users.
    
    \item \textbf{Notification Page:} It will display various types of notifications received from the platform. For example, students will
    see notifications about updates on their application status, companies will receive notifications about new applications, and 
    universities will be notified about new student registrations on the platform.

    \item \textbf{Side Menu:} The side menu is a navigation menu that provides access to the main sections of the platform, depends on the
    user's type. It will be displayed on the left side of the screen, with different sections available. For example, for students, the 
    side menu will include sections such as Search Internships, Dashboard, My Profile, My Applications, Internship History, Chat.

    \item \textbf{Header Bar:} The header bar is displayed at the top of the screen and includes the platform's logo, a menu icon, a chat 
    icon, a notification icon, and the user's profile icon. It remains visible on all pages once the user has logged in.

    \item \textbf{Chat Interface:} The chat interface is accessible from either the side menu or the header bar of the platform. It allows 
    students and companies to communicate with each other in real-time. They can view a list of available chats, send and receive messages, 
    and access their chat history.

    \item \textbf{Profile Page:} This interface is used by user to view and edit their profile data in case of necessary, in particular
    the student can update their CV accessing this page.

    \item \textbf{Search page:} This page contains a search bar and displays the results based on the user's search. For example, a student
    can search for internships, and the platform will return a list of results based on the student's input. 
    
    \item \textbf{Internship evaluation page:} This page is used by both students and companies to evaluate internship performance. Both 
    parties can provide feedback during the internship, and at the end, to complete the evaluation process, they must leave final comments 
    and rate the internship. Students also have the option to leave anonymous reviews about their experience, which can help other students
    make decisions about internships. These reviews will be visible when user visiting the company's profile.

\end{itemize}

Since the platform is designed for three types of users, the specialized interfaces for each user type are described in further detail
below, considering the characteristics of each user.
\begin{itemize}
    \item \textbf{Student Interface:}
    \begin{itemize}
        \item \textbf{Student Dashboard:} This is the home page that appears once the student logs in. On this page, the student gets an 
        overview of all the main functionalities they can access. The dashboard consists of:

        1. Search Bar with Filter Command: Allows the student to search and filter based on specific criteria.
        
        2. Overview of Suggested Internships: Displays job search keywords and internships recommended by the platform's algorithm.
        
        3. My Applications: Provides a quick overview of the status of the student's applications.
        
        4. Internship History: Displays a record of the internships the student has applied for in the past, including the internship they 
        are currently doing if exist.

        \item \textbf{My applications:} The student can track the status of their applications and view the details of the internships 
        they have applied for. Through this interface, the student can also receive information related to the selection process, such 
        as the interview date, the interview result, and the final decision of the company. Additionally, the student can filter their 
        applications based on their status, such as ``waiting''.

        \item \textbf{Internship history:} The student can view the internships they have participated in so far, along with the 
        evaluations related to their experiences. The page also allows the student to access the feedback section, where they can 
        leave comments or notes about a specific internship.

    \end{itemize}
    \item \textbf{Company Interface:} 
    \begin{itemize}
        \item \textbf{Company Dashboard:} It's the home page once the company logs in. On this page, the company gets an overview of 
        all the main functionalities they can access. The dashboard consists of:

        1. Search Bar with Filter Command: Allows the company to search and filter results based on specific criteria.
        
        2. Overview of Student Profiles: Displays student profiles suggested by the platform's matching algorithm for different internship 
        announcements.
        
        3. Internship Announcements: A list of internship announcements the company has published, along with the number of the applications
        received.
        
       \item \textbf{Publish Internship Page:} This is the main functionality of the company interface. On this page, the company can 
       create a new internship announcement and post it on the platform. The company is required to enter the necessary information, 
       following the guidelines provided by the platform.

        \item \textbf{Internship management page:} This page is divided into several sections: one for internship announcements that have 
        been published but are still in the publication phase, one for internships in the selection phase, one for internships currently 
        in progress, and one for closed internships. From this interface, the company can manage internships of all phases.
    \end{itemize}

    \item \textbf{University Interface:} 
    \begin{itemize}
        \item \textbf{University Dashboard:} The university dashboard displays a list of students and their associated activities. 
        The university can view the list of students registered on the platform and select a student to see detailed information about 
        their profile, activities, and internships they are currently doing or have completed in the past.
    \end{itemize}
\end{itemize}


\subsection{Hardware Interfaces}
Students\&Companies is a web-based platform, so it can be accessed from any device with an internet connection. The platform is
compatible with all modern web browsers. It is designed to be responsive and work on different screen sizes, including desktops, 
laptops, tablets, and smartphones. \\
The system will be hosted on multiple server that meet the requirements for web hosting. They will be responsible for the platform's 
backend processes, including the reccomendation algorithm, the data collection and the statistical analyses.


\subsection{Software Interfaces}
The system will interact with an emailing system to confirm the registration of new users. \\
It will also integrate various APIs to provide additional functionalities, for instance an API for implementing statistical analyses 
based on collected data that will be used to improve the recommendation algorithm. Another example would be an API to interact with
the database in order to store and retrieve information.


\subsection{Communication Interfaces}
The system will use HTTPS to ensure secure communication between the client and the server. The platform will also use WebSockets to
enable real-time communication between users, such as chat and notifications functionalities. Since the platform is designed to be a 
RESTful web application, it will use JSON as the data interchange format between the client and the server. \\
To interact with the mailing system, the platform will use SMTP protocol.


\section{Functional Requirements}
\subsection{Functional Requirement}
\subsection{Requirement Mapping}
\subsection{Use Case Diagram}
\subsection{Traceability Matrix}


\section{Performance Requirements}
\begin{itemize}
    \item \textbf{Concurrent access of users and resource utilization:} A platform like Students\&Companies is expected to have a large
    number of users, so it must be able to handle multiple requests simultaneously. By searching online for platforms that offer similar
    services we have found that a good estimate for the number of concurrent users that the platform should be able to manage is roughly
    a few thousands, peaking at a few tens of thousands during peak internship publishing periods (like at the end of universities semesters). \\
    The platform should be able to optimixe the resource utilization to ensure that the system can handle the load without any performance
    issues. This includes optimizing the database queries, caching data, and using load balancing techniques to distribute the load across
    multiple servers.
    
    \item \textbf{Data processing and storage:} The system should be able to process and store a large amount of data efficiently. From 
    what we were able to find online, similar platforms estimate that the number of registered users can reach a few hundreds of thousands,
    and the number of internships published can reach a few tens of thousands. For this reason, the database should be able to easily handle
    a few terabytes of data. \\
    The platform should also be able to process data quickly, especially when performing statistical analyses in order to generate recommendations 
    for students and companies. This includes optimizing the queries made by recommendation algorithm to ensure that it can provide 
    recommendations in real-time.

    \item \textbf{Availability:} The system should have a required uptime of at least 99.8\%, which means that the platform should be
    available 99.8\% of the time. This is equivalent to a downtime of less than 18 hours per year. To achieve this, the platform should be
    designed with fault tolerance in mind to ensure that the system can handle failures without affecting the availability. \\
    During the downtime period, a maintenance page should be displayed to inform users that the platform is being updated or is experiencing
    technical difficulties and is currently unavailable.

    \item \textbf{Time of Response:} From the users' perspective, the system should be responsive, meaning the response to any of his request 
    should appear instantly. In order to achieve this, the response time for most operations, such as loading a page, submitting an application, 
    or performing a search, should be at most a few seconds during peak usage and in the domain of milliseconds in normal conditions. \\
    Particular attention should be given to the recommendation algorithm, which should be able to provide recommendations in real-time,
    to the chat functionality, which should allow users to communicate in real-time, and to the notification system, who must ensure that
    updates are delivered to the user before relevant deadlines expire. \\
    The response time of other operations such as the ones that involve the mailing system cannot be guaranteed by S\&C.

\end{itemize}

\section{Design Constraints}

\subsection{Standards Compliance}
\subsection{Hardware Limitations}
\subsection{Any Other Constraint}

\section{Software System Attributes}
\subsection{Reliability}
\subsection{Availability}
\subsection{Security}
\subsection{Maintainability}
\subsection{Portability}
