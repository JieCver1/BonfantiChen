\renewcommand{\thesection}{\Alph{section}}
\section{Purpose}\label{sec:purpose}
The purpose of this document is to provide a detailed description of Student\&Companies. It will help developers to implement the 
required system features and it should provide the customer with a clear description of the system, allowing him to verify that it
meets the specified requirements.

\section{Scope}\label{sec:scope}
Student\&Companies is a platform that connects students, companies, and universities to facilitate the internship research, announcement and selection 
process. The platform provides services such as internship announcements, profile management, a recommendation system, interview management, and 
performance feedback. The platform is designed to be user-friendly and easy to use for students, companies, and universities. It is a web-based 
application that can be accessed from any device with an internet connection.

\section{Definition,acronyms,abbreviations}\label{sec:definition,acronyms,abbreviations}
\subsection{Definitions}\label{subsec:definitions}
\begin{itemize}
    \item \textbf{Student:} A person who is looking for internships.
    \item \textbf{Company:} An organization which wants to announce internship opportunities to students.
    \item \textbf{University:} An educational institution that is related to students and their internships.
    \item \textbf{User:} A generic term for students, companies, and universities who use the platform.
    \item \textbf{Candidate:} A term for students whose applications are selected and that will take part in the interview process.
    \item \textbf{Internship:} A opportunity offered by companies to students to gain practical experience in a real job environment.
    \item \textbf{CV:} Curriculum Vitae, a document that contains all necessary information about students to be able to apply for internships.
    \item \textbf{Recommendation:} A suggestion made by the platform to students and companies based on statistical analyses and keyword searches.
    \item \textbf{Interview:} A questionnaire form, that can be followed by an external meeting between students and companies, to evaluate the
    student preparation and make him understand what the company is looking for. 
    \item \textbf{Feedback:} Helpful information written by students and companies about their internship experiences to improve a performance 
    of the two parties.
    \item \textbf{Notification:} A message sent by the platform to inform students and companies about important events, such as new internship 
    offers, matching CVs, interview results etc.
    \item \textbf{Interview:} A meeting between students and companies to decide an assignment of the internship offer.
    \item \textbf{Platform:} The Students\&Companies (S\&C) system that provides the services to students, companies, and universities about
    internships.
    \item \textbf{Keyword:} A significant word or tag used to describe content, such as the skills, experiences, and preferences of students and 
    companies.
    \item \textbf{Comment:} The text that is written by students and companies to provide feedback or complaints about their internship experiences.
    \item \textbf{Complain:} A text that expresses dissatisfaction, issues, or annoyance about the internship experiences. It will be treated as a
    synonym of feedback in this document.
\end{itemize}

\subsection{Acronyms}\label{subsec:acronyms}
\begin{itemize}
    \item \textbf{S\&C:} Students\&Companies
    \item \textbf{CV:} Curriculum Vitae
    \item \textbf{GUI:} Graphical User Interface
    \item \textbf{API:} Application Programming Interface
    \item \textbf{HTTPS:} Hypertext Transfer Protocol Secure
    \item \textbf{REST:} Representational State Transfer
    \item \textbf{UML:} Unified Modeling Language
    \item \textbf{RASD:} Requirement Analysis and Specification Document
    \item \textbf{DMZ:} Demilitarized Zone
    \item \textbf{SMTP:} Simple Mail Transfer Protocol
    \item \textbf{DBMS:} Database Management System
    \item \textbf{SQL:} Structured Query Language
    \item \textbf{EIN:} Employer Identification Number
    \item \textbf{RDBMS:} Relational Database Management System
    \item \textbf{MVC:} Model-View-Controller
    \item \textbf{IDS:} Intrusion Detection System
\end{itemize}

\subsection{Abbreviations}\label{subsec:abbreviations}

\section{Revision History}\label{sec:revision history}
\begin{itemize}
    \item \textbf{Version 1.0} – 3/1/2025
\end{itemize}
\section{Reference Documents}\label{sec:reference documents}
\begin{itemize}
    \item Assignment RDD AY 2024–2025.
\end{itemize}

\section{Document Structure}\label{sec:structure}
This document is structured as follows:
\begin{itemize}
    \item \textbf{Section 1: Introduction} 
    \\It contains the summary of main architectural styles and choices that drive the design of the system. It also provides a brief description of the scope of the 
    project as already mentioned in the RASD. In addition, it include the specification of definitions, acronyms, and abbreviations of the terms used in this document. 
    At the end, it notes the revision history for updates to the document and the reference documents that were used during the development of this document.
    \item \textbf{Section 2: Architectural Design} 
    \\It provides a high-level overview of the system architecture, including the main components and their interactions. It also describes in detail each component
    by specifying the interfaces using the UML component diagram and list of the methods that each component can perform
    Then the deployment view of the system is presented with the description of the hardware and software components and the behavior of the system in runtime view 
    is presented with the sequence diagrams that illustrate the interactions. Finally the architectural styles and patterns chosen for the system will be 
    reported with a brief description of the reason for the choice.
    \item \textbf{Section 3: User Interface Design}
    \\In this section, the user interface design of the system is presented with a brief description of the interactions between each component and mockups that 
    illustrate the main functionalities of the system. With respect to the RASD, this section provides a more detailed description of the user interface design and 
    highlights the interactions from the perspectives of different types of users.
    \item \textbf{Section 4: Requirements Traceability}
    \\It connects the requirements of the system with the components of the architecture presented in the previous sections by providing a table that maps each
    requirement to the components that implement it. 
    \item \textbf{Section 5: Implementation, Integration, and Test Plan}
    \\Describe the plan for the implementation, integration, and testing of the system. It includes the description of the development process and emphasizes the
    the order of the implementation of the components withing the specification of the reason for the choice. It also includes the description of the testing
    process and the aim of each test case.
    \item \textbf{Section 6: Effort Spent}
    \\The time spent by each group member on each task will be registered in this section. It will be used to present the effort dedicated 
    by each member and to present the progress of the development of the project.
    \item \textbf{Section 7: References}
    \\ The other references that not include in the reference documents will be added in this section.
\end{itemize}