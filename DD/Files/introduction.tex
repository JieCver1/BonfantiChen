\renewcommand{\thesection}{\Alph{section}}
\section{Purpose}\label{sec:purpose}
The purpose of this document is to provide a detailed description of Student\&Companies. It will help developers to implement the 
required system features and it should provide the customer with a clear description of the system, allowing him to verify that it
meets the specified requirements.

\section{Scope}\label{sec:scope}
Student\&Companies is a platform that connects students, companies, and universities to facilitate the internship research, announcement and selection 
process. The platform provides services such as internship announcements, profile management, a recommendation system, interview management, and 
performance feedback. The platform is designed to be user-friendly and easy to use for students, companies, and universities. It is a web-based 
application that can be accessed from any device with an internet connection.

\section{Definition,acronyms,abbreviations}\label{sec:definition,acronyms,abbreviations}
\subsection{Definitions}\label{subsec:definitions}
\begin{itemize}
    \item \textbf{Student:} A person who is looking for internships.
    \item \textbf{Company:} An organization which wants to announce internship opportunities to students.
    \item \textbf{University:} An educational institution that is related to students and their internships.
    \item \textbf{User:} A generic term for students, companies, and universities who use the platform.
    \item \textbf{Candidate:} A term for students whose applications are selected and that will take part in the interview process.
    \item \textbf{Internship:} A opportunity offered by companies to students to gain practical experience in a real job environment.
    \item \textbf{CV:} Curriculum Vitae, a document that contains all necessary information about students to be able to apply for internships.
    \item \textbf{Recommendation:} A suggestion made by the platform to students and companies based on statistical analyses and keyword searches.
    \item \textbf{Interview:} A questionnaire form, that can be followed by an external meeting between students and companies, to evaluate the
    student preparation and make him understand what the company is looking for. 
    \item \textbf{Feedback:} Helpful information written by students and companies about their internship experiences to improve a performance 
    of the two parties.
    \item \textbf{Notification:} A message sent by the platform to inform students and companies about important events, such as new internship 
    offers, matching CVs, interview results etc.
    \item \textbf{Interview:} A meeting between students and companies to decide an assignment of the internship offer.
    \item \textbf{Platform:} The Students\&Companies (S\&C) system that provides the services to students, companies, and universities about
    internships.
    \item \textbf{Keyword:} A significant word or tag used to describe content, such as the skills, experiences, and preferences of students and 
    companies.
    \item \textbf{Comment:} The text that is written by students and companies to provide feedback or complaints about their internship experiences.
    \item \textbf{Complain:} A text that expresses dissatisfaction, issues, or annoyance about the internship experiences. It will be treated as a
    synonym of feedback in this document.
\end{itemize}

\subsection{Acronyms}\label{subsec:acronyms}
\begin{itemize}
    \item \textbf{S\&C:} Students\&Companies
    \item \textbf{CV:} Curriculum Vitae
    \item \textbf{UI:} User Interface
    \item \textbf{UX:} User Experience
    \item \textbf{API:} Application Programming Interface
    \item \textbf{HTTPS:} Hypertext Transfer Protocol Secure
    \item \textbf{TLS:} Transport Layer Security
    \item \textbf{REST:} Representational State Transfer

\end{itemize}

\subsection{Abbreviations}\label{subsec:abbreviations}
\begin{itemize}
%    \item \textbf{[Gn]:} Used to number the goals, where Gn is the n-th goal.
%    \item \textbf{[Pn]:} Used to number the phenomena, where Pn is the n-th phenomenon.
%    \item \textbf{[Rn]:} Used to number the requirements, where Rn is the n-th requirement.
%    \item \textbf{[An]:} Used to number the domain assumptions, where An is the n-th assumption.
    \item \textbf{CO:} Company
    \item \textbf{ST:} Student
    \item \textbf{UNI:} University
\end{itemize}

\section{Revision History}\label{sec:revision history}

\section{Reference Documents}\label{sec:reference documents}
\begin{itemize}
    \item Assignment RDD AY 2024–2025.
\end{itemize}

\section{Document Structure}\label{sec:structure}
This document is structured as follows:
\begin{itemize}
    \item \textbf{Section 1: Introduction} 
    \\ 
    \item \textbf{Section 2: Architectural Design}
    \\ 
    \item \textbf{Section 3: User Interface Design}
    \\
    \item \textbf{Section 4: Requirements Traceability}
    \\
    \item \textbf{Section 5: Implementation, Integration, and Test Plan}
    \\
    \item \textbf{Section 6: Effort Spent}
    \\The time spent by each group member on each task will be registered in this section. It will be used to present the effort dedicated 
    by each member and to present the progress of the development of the project.
    \item \textbf{Section 7: References}
    \\ The other references that not include in the reference documents will be added in this section.
\end{itemize}